\documentclass[a4paper, 12pt]{article}
\usepackage[top=2cm, right=3cm, bottom=2cm, left=3cm]{geometry}

\usepackage[utf8]{inputenc}
\usepackage[T1]{fontenc}
\usepackage[portuguese]{babel}

\usepackage{amsmath, amsfonts, amssymb}
\usepackage{hyperref}

\newcommand{\tbf}[1]{\textbf{#1}}

\begin{document}
	\begin{itemize}
		\item\tbf{Nome: Renan da Silva Guedes}
		\item\tbf{RA: 223979}
	\end{itemize}

	\begin{enumerate}
		\item\tbf{Qual motivo que os livros de hidrologia tratam o dimensionamento de reservatórios de água como regularização de vazões por reservatórios?}
		
		O dimensionamento de reservatórios está intimamente ligado a capacidade dos mesmos de armazenar água para épocas mais secas. A vazão dos reservatórios é vinculada às condições climáticas de um local, ou seja, em épocas mais úmidas as vazões deles tende a aumentar, enquanto que em épocas mais úmidas a mesma reduz. Dessa forma, o dimensionamento dos reservatórios é tratado como regularização de vazões tendo em vista que o projeto feito de forma correta, contemplando todas as variáveis da região, não superdimensiona os reservatórios, aumentando os custos materiais e nem subdimensona os mesmos, ocasionando maior necessidade de racionamento em períodos de estiagem.
		
		\item\tbf{Quais são as principais informações necessárias para o dimensionamento do volume útil de reservatórios?}
		
		Devem ser consideradas as perdas por evaporação, infiltração no solo ou vazamentos, caso eles forem significativos.
	\end{enumerate}

	\section{Referências}
	
	LOPES, João; SANTOS, Raquel. Capacidade de Reservatórios. Métodos de dimensionamento, [S. l.], p. 8-10, 11 jul. 2002. Disponível em: http://www.leb.esalq.usp.br/leb/disciplinas/Fernando\newline /leb1440/Aula\%206/Capacidade\%20de\%20Reservatorios.pdf. Acesso em: 23 jun. 2020.
\end{document}