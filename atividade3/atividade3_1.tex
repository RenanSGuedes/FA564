\documentclass[a4paper, 12pt]{article}
\usepackage[utf8]{inputenc}

\usepackage{indentfirst}
\usepackage[T1]{fontenc}
\usepackage{amsmath, amssymb, amsfonts}
\usepackage[top=3cm, left=2.5cm, right=3cm, bottom=3cm]{geometry}
\usepackage{fontspec}
\setmainfont{Arial}
\usepackage{hyperref}

\begin{document}
    \begin{itemize}
        \item\textbf{Nome: Renan da Silva Guedes}
        \item\textbf{RA: 223979}
    \end{itemize}
    
    \begin{center}
        \begin{large}
            \textbf{Estudo dirigido 1}
        \end{large}
    \end{center}
    
    \begin{enumerate}
        \item[1] Ao acumular água a partir da construção das barragens, os reservatórios desempenham o papel de possibilitar aos grupos humanos a gestão da água segundo a demanda requisitada. 
        
        \item[2] Os reservatórios de acumulação de água desempenham múltiplos papéis. Dentre eles, é visado principalmente sempre manter disponível a água para o consumo humano, independente da época.
        
        \item[3] O dimensionamento adequado deve ser feito visando contemplar todas as atividades as quais ele será exigido. Dessa forma, o volume é de fundamental importância para conseguir suprir a demanda de todas as atividades.
        
        \item[4.1] Providenciar energia elétrica para as residências de modo que a sociedade consiga desempenhar suas atividades cotidianas.
        
        \item[4.2] Ao reter a água localmente torna-se possível seu consumo doméstico de diferentes formas e 
        principalmente o abastecimento local em épocas de maior aridez ou na necessidade de realizar manutenções no curso hídrico.
        
        \item[4.3] Ao fornecer água para os sistemas de abastecimento é facilitado o acesso a essa fonte hídrica sob diferentes condições climáticas, temporais ou para situações onde se exige a manutenção do sistema, como dito anteriormente. Ou seja, o abastecimento é uma forma de atingir a estabilidade ao sempre disponibilizar água sob condições ideais.
        
        \item[4.4] Por meio da recreação possibilita-se a interação social de diferentes grupos humanos e, dessa forma, uma maior aproximação do ambiente natural e distanciado da crescente urbanização atual. Além de possuir um importante papel econômico à sociedade.
        
        \item[4.5] Com base nessa prática torna-se possível suprir a demanda por alimentos da sociedade. Ao desconcentrar os locais de produção é possível fornecer o produto com maior rapidez e qualidade sem estar sujeito às adversidades do modal de transportes rodoviário, por exemplo.
        
        \item[4.6] Ao fazer a proteção a jusante de uma barragem é visado reduzir o impacto que as inundações podem trazer às regiões próximas que podem ser afetadas em diversos aspectos. A proteção também desempenha o papel de conter o avanço de efeitos erosivos na parede da barragem que recebe a água proveniente da montante perdendo energia ao ir para uma menor altitude. Dessa forma, a proteção é importante de modo a tornar a estrutura não colapsível. 
        
        \item[4.7] O armazenamento é indispensável tendo em vista que regiões mais áridas sofrem com a insuficiência de água e, dessa forma, ao fazer seu abastecimento busca-se garantir água em épocas turbulentas.
        
        \item[4.8] No quesito navegação, é fundamental para o transporte hidroviário de pessoas ou produtos de um lugar para o outro, sendo uma alternativa aos países que empregam mais essa modalidade quando comparado aos modais de ferrovia, rodovias ou aeroviários.
        
        \item[4.9] Tal controle é importante para garantir a segurança no uso da água para diversos fins. Ou seja, deve ser buscado não afetar a qualidade da água dos reservatórios a partir de uma boa infraestrutura a montante da barragem, impedindo que resíduos tóxicos afetem a flora, fauna ou os grupos humanos que farão uso dos corpos d'água provenientes.
        
        \item[5.1] O assoreamento deve ser considerado visando impedir impactos sobre a população próxima aos reservatórios sujeitos ao acúmulo e deposição de sedimentos.
        
        \item[5.2] Os dois parâmetros são relevantes ao analisar a importância que a água subterrânea desempenha em nível global. Devido à grande parte da água doce do planeta estar contida no subsolo, é fundamental o pensamento conservacionista visando dirimir o impacto sobre esses locais já que eles são responsáveis pela manutenção de grande parte das florestas de climas tropicais ou seco. Logo, preservar essas águas não impacta somente em sua disponibilidade, mas também na biodiversidade de diferentes regiões. 
        
        \item[5.3] Ao alterar o clima de uma região é imprescindível a tomada de medidas por parte da sociedade, tendo em vista que ela pode estar sujeita a novos fenômenos naturais que possam impactar negativamente a depender da infraestrutura local da região diante das adversidades naturais. 
        
        \item[5.4] Esse fator é relevante tendo em vista a área de risco onde essa populações habitam. Dessa forma, exigisse desses indivíduos o preparo adequado para situações de emergência quanto a ruptura desses reservatórios. Logo, torna-se cada vez mais comum a necessidade de se estar atento aos alertas feitos via sirene de modo a executar a evacuação do perímetro o mais rápido para garantir a segurança. Porém, grande parte dos grupos humanos habitantes dessas regiões não recebem as devidas instruções quanto as medidas a serem tomadas e, assim ficam mais suscetíveis às adversidades.
        
        \item[5.5] Tendo em vista a maior susceptibilidade de 
        populações que habitem regiões endêmicas para doenças associadas aos corpos d'água, são necessárias medidas de conscientização dos indivíduos visando a melhorar as práticas que conduzam a uma menor incidência das doenças. 
        
        \item[5.6] Sem dúvida ao construir um reservatório a biodiversidade aquática é afetada. Por exemplo, muitas espécies de peixes são comprometidas por conta das mudanças ocorrentes no fluxo hídrico do rios. Devido a isso, são impedidos de desovar nos locais de costume, sendo necessário maior adaptação visando estender a vida da espécie. Não só a parte biológica é afetada como é visto na fauna anteriormente, as propriedades químicas da água podem sofrer a ação do reservatórios deixando de transportar quantidades de gases e nutrientes ideais para a sobrevivência da macro e microfauna aquática.
        
        \item[5.7] Ao limitar áreas para a construção dos reservatórios os impactos à fauna e flora são diversos. Muitas espécies pertencentes à fauna podem sofrer devido à maior restrição de espaço ocorrente, sendo forçadas a migrar para regiões muitas vezes insalubres de sobrevivência. A flora por sua vez, sofre com a crescente remoção de sua área visando contemplar projetos de grande magnitude, podendo ocasionar a redução dessa biodiversidade.
        
        \item[5.8] Esse eventos são importantes de serem levados em consideração tendo em vista que o avanço de processos erosivos na margem dos rios pode alcançar localidades periféricas podendo ameaçar a estabilidade desses locais. Dessa forma, são necessárias medidas conservacionistas visando frear tais fenômenos. Além disso, ao agravar os efeitos erosivos o nível das águas pode subir podendo ser uma ameaça em épocas com maiores taxas pluviométricas.
        
        \item[5.9] Tal fator é importante ao considerar a fragilidade do material no qual a barragem está estabelecida. Dessa forma, ao ser submetida a altas pressões a superfície terrestre tende a ser compactada. Por conta disso, é fundamental garantir que o material abaixo dos reservatórios está adequadamente compactado e se o material de origem é adequado visando garantir as características mecânicas exigidas para suportar altas pressões sem deformar, trincar ou comprometer o projeto durante ou após sua finalização.
        
        \item[5.10] Ao saber do papel que cada uma dessas regiões desempenham na sociedade, desde a produção de alimentos, turismo ou de habitação é importante que o pensamento coletivo esteja baseado no uso consciente de tais localidades de modo a permitir as ações de cunho social, cultural e econômico a plenitude em todos os setores.  
        
        \item[5.11] Tal fator é relevante levando em conta a ameaça que a ruptura de barragens representa para a sociedade ao impactar negativamente o solo, a biodiversidade local, às vidas humanas ou economicamente tendo em vista a escalabilidade de danos irreparáveis.
    \end{enumerate}
	\newpage

	\section{\textbf{Referências}}

	\noindent ÁGUA, BLOG: ÁGUAS SUBTERRÂNEAS: O QUE É E QUAL A IMPORTÂNCIA?. [S. l.], 21 jun. 2019. Disponível em: https://www.eosconsultores.com.br/aguas-subterraneas/. Acesso em: 8 jun. 2020.
	
	\vspace{.5cm}
	\noindent RIBEIRO, Admilson. \textbf{A avaliação de impactos ambientais e as barragens de rejeitos}: Docente da Unesp de Sorocaba ressalta a importância de estudos ambientais na mineração. [S. l.], 12 fev. 2019. Disponível em: https://www2.unesp.br/portal\#!/noticia/34275/a-avaliacao-de-impactos-ambientais-e-as-barragens-de-rejeitos/. Acesso em: 3 jun. 2020.
	
	\vspace{.5cm}
	\noindent RODRIGUES, Alex. \textbf{Risco de rompimento de barragem obriga moradores a deixarem suas casas}: Empresa responsável pela barragem no ES está monitorando a situação. Brasília, 27 jan. 2020. Disponível em: https://agenciabrasil.ebc\newline.com.br/geral/noticia/2020-01/risco-de-rompimento-de-barragem-obriga-moradores-deixarem-suas-casas. Acesso em: 4 jun. 2020.
\end{document}