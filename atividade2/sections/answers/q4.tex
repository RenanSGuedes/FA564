\item[]\textbf{Questão 4}

\hspace{1cm} O solo é compactado visando melhorias na qualidade mecânica a partir da redução do índice de vazios.

\hspace{1cm} O processo de compactação se dá no solo. O solo é composto por três elementos essenciais: ar, água e material sólido. Dessa forma, para que haja a compactação de uma amostra deve-se reduzir o teor de água e ar na porção. Todavia, para se realizar tal procedimento pode ser feito uso de duas categorias de compactação. A primeira, classificada como mecânica, é realizada aplicando máquinas, enquanto a segunda -- manual -- é realizada pelo ser humano ao utilizar ferramentas para assistência. 

\hspace{1cm} O ensaio utilizado para se obter os parâmetros de compactação foi idealizado por Ralph Proctor (1933) recebendo seu sobrenome. Nele é feito uso de um cilindro metálico e uma amostra de solo a ser analisada. A primeira etapa consiste na secagem e destorroamento da amostra, de modo que o material se torne homogêneo. Antes de inseri-lo no cilindro é necessário deixar a amostra com \SI{5}{\%} de sua umidade abaixo da ótima. Dessa forma, é feita a sua inserção no cilindro seguida do amassamento até atingir uma altura próxima da terça parte da altura do cilindro. O procedimento é repetido para a segunda e terceira camada, sempre acrescentando a amostra de modo a ocupar um terço do volume do cilindro. Para a última camada é importante utilizar um anel que garanta homogeneidade da compactação do material eminente. Por fim, para que a altura de solo coincida com a do cilindro é retirado o excesso por meio de uma espátula. Com o cilindro totalmente preenchido, leva-se o conjunto solo mais cilindro para a pesagem numa balança. Com base na massa obtida e volume do cilindro ($\approx\SI{1}{\deci\meter^{3}}$), calcula-se a massa específica da amostra. Por fim, com esse dado obtém-se a massa específica seca. Feito isso, com o valor de umidade e massa específica da amostra seca é feita a plotagem gráfica posicionando a coordenada obtida. Basta repetir o processo aumentando a umidade em \SI{2}{\%} e coletando a nova massa específica seca. Com isso, ao notar queda nos valores da massa específica é possível conseguir a umidade ótima da amostra. Nesse ponto ocorre o melhor estabelecimento da compactação sendo o ponto ideal para conferir boa qualidade mecânica ao solo. Entretanto, é importante lembrar que o ensaio de Proctor é válido para valores de umidade e massa específica seca obtidos para uma energia de compactação constante.