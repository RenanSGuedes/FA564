\item[]\textbf{Questão 8}


\hspace{1cm}Os sistemas de drenos são construídos visando controlar a erosão interna das barragens. Dessa forma, é feita a inclusão de filtros, drenos e transições subverticais ou inclinadas na obra.

\hspace{1cm}Normalmente são utilizadas areias com granulometria previamente estabelecida, as quais devem ser devidamente compactadas durante a execução. O dreno vertical geralmente é composto por areia grossa, aluvionar isenta de finos, sendo especificado que no máximo 5\% do material atravesse um peneira \#200. Esse procedimento é realizado visando impedir a coesão que pode originar trincas de tração no interior do dreno. Dessa forma, devem ser satisfeitas as condições de filtragem e drenagem da água percolada através da barragem, ou seja, os seus vazios devem ser suficientemente pequenos, para evitar que as partículas do aterro sejam carreadas através deles e suficientemente grandes, para proporcionar permeabilidade adequada para o escoamento da água, evitando o desenvolvimento de elevadas forças de percolação e de pressões hidrostáticas (Gaioto, 2003) 