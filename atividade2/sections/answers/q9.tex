\item[]\textbf{Questão 9}

\hspace{1cm}Para a realização desse procedimento denominado de fase dos desvios de rios algumas características devem ser levadas em consideração, como:

\begin{enumerate}
	\item Características da obra
	\item Condições topográficas
	\item Condições geológicas
	\item Cronograma da obra (plano de execução)
\end{enumerate}

\hspace{1cm}As fases referidas podem ser divididas em duas categorias. A primeira diz respeito aos desvio contendo somente uma fase, enquanto a segunda apresenta múltiplas fases.

\hspace{1cm}Os fatores que são condicionantes na escolha do desvio do rio influem no tipo de fase escolhido e nas estruturas de desvio a serem utilizadas. Dessa forma, são três os condicionantes para a execução dessa obra: 

\begin{itemize}
	\item Físicos
	\item Técnicos
	\item Financeiros
\end{itemize} 

Podendo ser estendidos às seguintes etapas

\begin{itemize}
	\item Aspectos físicos:
	\begin{itemize}
		\item Topografia
		\item Geologia
		\item Regime hidrológico e hidráulico; e
		\item Localização.
	\end{itemize}
	\item Aspectos técnicos:
	\begin{itemize}
		\item Características da obra principal:
		\begin{itemize}
			\item Arranjo geral;
			\item Cronograma da obra; e
			\item Métodos e materiais construtivos.
		\end{itemize}
		\item Impacto sócio-ambiental;
		\item Experiência da projetista e da construtora;
		\item Reaproveitamento de equipamentos e estruturas de obras anteriores; e
		\item Risco de falha aceitável.
	\end{itemize}
	\item Aspectos financeiros
	\begin{itemize}
		\item Custos das obras.
	\end{itemize}
\end{itemize}