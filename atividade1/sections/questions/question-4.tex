
A causa dos atoleiros é devido a falta de capacidade de suporte do subleito e ausência ou deficiência do sistema de drenagem. Para corrigir esse problema deve-se iniciar com a retirada de água acumulada no local através de valetas e sangras. Em seguida, coloca-se uma camada de reforço. Sobre esta, executa-se o revestimento primário ou então o agulhamento.

A pista seca derrapante é proveniente do ``encascalhamento'' de material granular de qualquer dimensão sem ligante (argila). Pode aparecer também em terrenos onde o leito natural é formado por material granular ou pedras pequenas, ou através de deterioração de um tratamento primário mal executado, pobre em ligante (argila).
A correção baseia-se na análise do material granular. Se o mesmo for grosseiro e o leito rochoso, pode-se realizar um agulhamento.