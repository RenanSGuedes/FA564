
Para a ampliação de uma estrada vicinal alguns parâmetros essenciais devem ser contemplados na análise. Tendo em vista a necessidade de expandir a obra existente alguns fatores podem ter sido relevantes para essa tomada de decisão. Muitas vezes o fluxo automobilístico influencia na necessidade de ampliação já que a requisição àquela rota é aumentada. Dessa maneira, sabendo do papel que essas estradas exercem ao conectar o perímetro rural ao urbano devem ser atendidos os parâmetros em ambas as localidades.

O primeiro fator está vinculado aos estudos geológicos e geotécnicos presentes no terreno a ser ampliado. Ou seja, é fundamental saber se o traçado escolhido para a estrada atende às condições como solo ideal, se há presença de solos moles (afetando de forma negativa), locais de erosão ou se é fonte de material rochoso eminente, por exemplo. No quesito solo, por se tratar de uma estrada sem pavimentação, é importante analisar o tipo presente no trecho a ser ampliado e verificar se o mesmo destoa do existente quando as condições físicas oferecidas. Assim, é possível buscar fazer  prévias correções caso a área apresente excessiva erosão ou uma concentração de partículas de areia, silte e argila que tornam a estrutura colapsível.

Outro condição diz respeito à profundidade do leito rochoso. É importante garantir que o perfil de solo que comporá a estrada tenha um espessura adequada visando impedir a proeminência de estruturas rochosas que vão exigir manutenção. 

A porosidade do solo é relevante na análise, pois alguns trechos da estrada vicinal, ao não estarem compactados, são mais suscetíveis à deformação e transporte de material erodido. Dessa forma, o estudo geotécnico é pertinente nesse quesito ao viabilizar, por meio de sondagens, previsões do perfil do solo e quais as vulnerabilidades e riscos que ele pode correr futuramente.

Outras questões que devem entrar no orçamento dizem respeito ao projeto geométrico, obras de terra e a drenagem. Ou seja, após garantir as condições físicas estruturais é fundamental planejar o local que a estrada será estabelecida e traçar a rota de modo a evitar pontos cegos. Contudo, deve-se levar em conta nesse percurso se haverá necessidade de desapropriação de moradias e se há viabilidade para a execução desses planos. Feito isso, as obras de terra terão o papel de remover o material presente na região e destiná-lo a um local adequado para a deposição. 

Realizada a remoção de terra, como a estrada não exige pavimentação é cabível iniciar o dimensionamento do sistema de drenagem do local, já que o acúmulo de água pode aumentar a erosão e a formação de possas na estrada.

Para garantir a segurança do tráfego, é imprescindível inserir a sinalização adequada nos trechos da estrada. Muitas vezes, na região onde será estabelecida a estrada pode haver presença de animais silvestres. Estes, por sua vez, podem trazer riscos ao tráfego local ao serem atropelados. De modo a dirimir os riscos trazidos a ambas as partes, é válida a inserção de cercas ao redor da estrada nos trechos de maior incidência de acidentes. Porém, a cerca deve conter maior aproximação dos seus segmentos, caso ela seja feita de madeira e arame. Dessa forma, torna-se mais fácil diminuir o fluxo de animais de vários portes. 
 
Um quesito essencial que não deve ser esquecido está associado ao projeto de obras de arte. Caso exista um rio no caminho e o mesmo exija ser atravessado pela estrada, será fundamental a construção de uma ponte, por exemplo. Sabendo que esse tipo de projeto contempla a transposição de obstáculos é cabível a análise de tal necessidade.

Por fim, deve-se levar em consideração o impacto ambiental trazido pelo conjunto de obras e verificar se ele atende os requisitos. Assim, é possível equilibrar e atender as necessidades do tráfego local e não degradar a fauna e flora presentes no perímetro da obra a ser realizada.